\documentclass[slidestop,compress,mathserif]{beamer}
\usepackage[latin1]{inputenc}
\usepackage{verbatim}
\usepackage{graphicx}
\usetheme{Boadilla}
\usecolortheme{beaver} %{beetle}%{crane} 
\usepackage{textpos}
\usepackage{tikz}


\title[WW Scattering Update]{$W^+W^-$ Scattering: Comparision of Two Different Methods for Polarized Sample Generation}

\author[Ramkrishna Sharma]{{\bf Ramkrishna Sharma}\inst{1}, Arun Kumar\inst{2}, Patricia Rebello Teles\inst{3}, Md. Naimuddin\inst{1}, Nhan V. Tran\inst{4}}
\institute[Delhi,INDIA]{\inst{1}University of Delhi, \inst{2}National Taiwan University, \inst{3}Brazilian Center for Physics Research, \inst{4}Fermi National Accelerator Lab.}
\date[June 12, 2015]{SMP-VV Meeting June 12, 2015}

\setbeamertemplate{footline}[slide number]
\setbeamertemplate{frametitle}[default][center]         % For centering the Heading

\titlegraphic{
	\includegraphics[width=1.5cm,keepaspectratio]{cern.jpg}\hspace*{3.35cm}~%
	\includegraphics[width=2cm,keepaspectratio]{logo_du.jpeg}\hspace*{2.75cm}~%
	\includegraphics[width=1.5cm,keepaspectratio]{cmsLogo.jpeg}
}

\begin{document}
\renewcommand{\inserttotalframenumber}{\pageref{lastslide}}
\begin{frame}
\titlepage
\end{frame}

%\begin{frame}\frametitle{Table of contents}\tableofcontents
%\end{frame}
%%%%%%%%%%%%%%%%%%%%%%%%%%%%%%%%%%%%%%%%%%%%%%%%%%%%%%%%%%%%%%%%%%%%%%%%%%%%%%%%%%%%%%%%%%%%%%%%%%%%%%%
\section{Introduction}
\begin{frame}\frametitle{Recap\ldots}
  \vfill
  \begin{itemize}
    \item Efficiency Generation Problem: \\{\color{blue} \url{https://indico.cern.ch/event/361122/contribution/4/material/slides/0.pdf}}
  \vfill
    \item Comparison of Polarized samples: \\{\color{blue} \url{https://indico.cern.ch/event/382113/contribution/4/material/slides/1.pdf}}
  \vfill
    \item Comparison of MadSpin \& DECAY package: \\{\color{blue}\url{https://indico.cern.ch/event/385528/contribution/2/material/slides/0.pdf}}
  \vfill
  \end{itemize}
\end{frame}

\section{Generation Method}
\begin{frame}\frametitle{Method-I: Recomended by MadGraph Authors}
  \vfill
  Ref.: {\color{blue} \url{https://answers.launchpad.net/mg5amcnlo/+faq/2243}}
  \vfill
  \begin{itemize}
    \item Generate $p p > w^+w^- j j$
  \vfill
    \item Edit files $SubProcesses/P^*/matrix^*.f$
  \vfill
    \item Launch the generation of Events 
  \vfill
  \end{itemize}
  {\color{red}Problem with this method: Generation efficiency for sample having both w's longitudinal polarized is only 5\%}. For other two process its 100\%.
  \vfill
  \begin{itemize}
    \item Then decay it using the DECAY package.
  \end{itemize}
  \vfill
\end{frame}

\begin{frame}\frametitle{Method-II: Suggested by Pietro \& Andrey (1/2)}
  Ref.: {\color{blue} \url{https://twiki.cern.ch/twiki/bin/view/Main/MadgraphPolarization}}
  \begin{itemize}
    \item Generate $p p > w^+w^- j j$ with latest version of MadGraph.
    \item Split it into 3 parts based on polarization. (By modifying lhe\_parser.py code in MadGraph)
      \begin{itemize}
	\item LL (approx. 5.5 \%),
	\item LT (approx. 34.9 \%), and
	\item TT (approx. 59.5 \%)
      \end{itemize}
    \item Decay them using DECAY package ({\color{red} Not supported by MadGraph anymore}).
      \begin{itemize}
	\item Two problems faced while decaying:
	  \begin{itemize}
	    \item decay script is not working: Older LHE file has not any blank line while new LHE files have blank lines. Fixed by Andrey (\url{https://answers.launchpad.net/mg5amcnlo/+question/257782})
	    \item Only able to decay one particle but not able to decay next particle: Because of format of $<init>$ info. (discussion link: \url{https://hypernews.cern.ch/HyperNews/CMS/get/progQuestions/340/1/1/1/1.html}). {\bf Thanks Tim Cox for help.}
	  \end{itemize}
      \end{itemize}
  \end{itemize}
\end{frame}


\begin{frame}\frametitle{Method-II: Suggested by Pietro \& Andrey (2/2)}
  \vfill
  \begin{itemize}
    \item Generate sample: $ p p > w^+w^- j  j$
  \vfill
    \item Split into 3 parts: LL, LT, and TT.
  \vfill
      \begin{itemize}
	    \item Again split each into two halves.
  \vfill
	\end{itemize}
    \item First file decay $w^+->l^+\nu$ and $w^-->jj$, and \\for 2nd file decay $w^+->jj $ \& $w^-->l^-\nu$
  \vfill
    \item Again merge both decay mode.
  \end{itemize}
  \vfill
\end{frame}

\begin{frame}\frametitle{Which Method One Should Use?}
  \begin{description} test
    \item [Method-I] 
      \begin{itemize}
	\item Efficiency problem for sample LL.
	\item Only $\approx5\%$ of samples generated out of demended.
	\item Need to run one script to modifying all $matrix^*.f$ files.
      \end{itemize}
    \item [Method-II] test
      \begin{itemize}
	\item Generation efficiency is 100\%
	\item one or two scripts to seperate different polarization
	\item Number of events: for LL 5\%, LT 34.9\%, and TT 59.5\% of total demended.
	\item To generate 0.1M events of LL we have to generate $\apprx$0.6M events of LT and $\approx$1.1M events of TT.
      \end{itemize}
  \end{description}
\end{frame}

\section{Plots}

\section{Summary}
\label{lastslide}
\begin{frame}\frametitle{Summary \& Conclusion}
  \begin{itemize}
    \item Results from both method are similar.
  \end{itemize}
\end{frame}


%	\begin{frame}\frametitle{}	\end{frame}

%	\begin{frame}\frametitle{}	\begin{itemize}		\end{itemize}	\end{frame}

%	\begin{itemize}		\end{itemize}

%	\begin{block}{ }	\end{block}

%	\begin{columns}[t]	\column{.5\textwidth}	\column{.5\textwidth}	\end{columns}

%	\includegraphics[width=12cm,height=8cm]{}
%%%%%%%%%%%%%%%%%%%%%%%%%%%%%%%%%%%%%%%%%%%%%%%%%%%%%%%%%%%%%%%%%%%%%%%%%%%%%%%%%%%%%%%%%%%%%%%%%%%
%%%		Put figure one above other to point something in next click
%\begin{frame}\frametitle{Mass of Two tagged Jet }
%	 \begin{tikzpicture}[every node/.style={anchor=center}]
% 		\node(a) at (8,6){\includegraphics[width=12cm,height=7cm]{Massesjj.pdf}};
%		\pause
%		\node(b) at (10,7){\includegraphics[width=6cm]{Massesjj_zoomed.pdf}};
%		\draw[red,thick,->](8,8)--(3.5,6.5);
%	 \end{tikzpicture}
%\end{frame}
%%%%%%%%%%%%%%%%%%%%%%%%%%%%%%%%%%%%%%%%%%%%%%%%%%%%%%%%%%%%%%%%%%%%%%%%%%%%%%%%%%%%%%%%%%%%%%%%%%%
%%%%%%%%%%%%%%%%%%%%%%%%%%%%%%%%%%%%%%%%%%%%%
%	WRITE SOMETHING ON FIGURE
%\begin{tikzpicture}
%\draw (0, 0) node[inner sep=0] 
%{\includegraphics[width=12cm,height=9cm]{W_muon_pt.pdf}};
%\draw (2, 3) node {\color{red} Comment On Plot};
%\end{tikzpicture}
%%%%%%%%%%%%%%%%%%%%%%%%%%%%%%%%%%%%%%%%%%%%%
%%%%%%%%%%%%%%%%%%%%%%%%%%%%%%%%%%%%%%%%%%%%%
%\begin{frame}[shrink]\frametitle{Number of events after each applied cut}
%{\scriptsize \begin{tabular}{|p{1.5cm}|p{1cm}|c|c|c|c|c|}
%    	\hline
%	Sample & Total Event  & muon $p_T, \eta$ & pfMET & $Jet_{|\eta|}$ &  $Jet_{p_T}$  & $dR>0.5$\\
%	\hline
%	\hline
%	signal & 8 & 8 & 7 & 7  & 7 & 7\\
%	\hline
%	Data/mc & 0.6210 & 0.7379 & 0.7263 & 0.8272 & 0.8512 & 0.8512\\
%	\hline
%\end{tabular}
%}
%\end{frame}
%%%%%%%%%%%%%%%%%%%%%%%%%%%%%%%%%%%%%%%%%%%%

\begin{frame}[c]
	\begin{center}
	\Huge Thanks
	\end{center}
\end{frame}

\begin{frame}[c]
	\begin{center}
	\Huge Backup Slides
	\end{center}
\end{frame}



\end{document}
